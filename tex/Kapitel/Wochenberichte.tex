\section{Wochenberichte}
\subsection{KW14 31.03-2025 - 04.04.2025: Erste Woche}
Am ersten Tag wurde ich herzlichst von meinem Betreuer Dominik Ulrich an der Hauptpforte des \ac{DCM} (Digital Campus Munich) Empfangen um von dort aus direkt zum \ac{FIZ} zu laufen und meinen Werksausweis abzuholen. 
Nachdem ich mich daraufhin den anwesenden persönlich vorgestellt hatte, bekam ich meinen Laptop und ein Arbeitshandy. 
Die erste Aufgabe bestand dann darin das Onboarding durchzuführen und die Arbeitsgeräte, sowie alle Accounts und Accessanfragen aufzusetzen und zu bearbeiten. 
Leider war das leichter gesagt als getan und schon das Setzen eines Initialpassworts hatte durch die komplexen Sicherheitsauflagen lange Zeit in Anspruch genommen und nach mehrfachen fehlschlagen einen Besuch bei der In-House IT mit sich geführt. 
Zwischenzeitlich standen schon die ersten interessanten Meetings an, in denen ich die restlichen Kollegen der Abteilung kennenlernen und mich selbst Vorstellen durfte. 
In den folgenden Tagen durfte ich mich mit ein paar Slides im monthly meeting der ganzen Abteilung (bei BMW auch Domain genannt) vorstellen sowie an der BMW internen Townhall zur Vorstellung der „Neuen Klasse“ teilnehmen. 
Gegen Ende der Woche hatte ich den Großteil der Geräte und Accounts eingerichtet und von Dominik, sowie meinem Kollegen Christian Bachmeier die \ac{NCPA} Applikation, mit welcher ich mich den Großteil des Praktikums viel beschäftigen werde, vorgestellt bekommen. 
In mehreren 1:1 Gesprächen mit Dominik wurden mir dann nach und nach auch die Unternehmensstrukturen, sowie weitere wichtige Informationen erklärt und darüber hinaus auch schon der erste Task zugeteilt. 
Dieser bestand aus Besprechungen über die Standardization des Tech-Stacks für \ac{CBS} und das Erstellen eines damit einhergehenden Onepagers auf dem dieser Tech Stack und die dazugehörigen Best Practises aufgelistet sind.


\subsection{KW15 07.04-2025 - 11.04.2025: UAT \& Strategy Event}
Die zweite Woche startete mit einem großen Workshop am Montag, in welchem das \ac{UAT} für das \ac{NCPA} mit 8 verschiedenen Märkten (um genau zu sein die Märkte aus Wave 4 – die anderen Waves wurden vor meiner zeit durch das \ac{UAT} geführt) durchzuführen. 
Hierbei stand ich Dominik bei dem Dänischen, Slowakischen und dem Schweizer Markt zur Seite und unterstützte ihn bei der Durchführung der einzelnen Test-Cases mit den jeweiligen Märkten. 
Durch das Testing Event kamen glücklicherweise nur kleine sogenannte „Defects“ auf, welche kleine Fehler im Programm beschrieben, die dann an das Developer Team reported werden müssen. 
Nachdem ich die Defect-tickets geschrieben hatte, wurde Dominik bewusst, dass die nicht von uns erstellten Defects keinem Dokumentationsschema folgen, weshalb ich am Folgetag dann ein kleines Template für das korrekte Defect reporting erstellte.

Gegen Mitte der Woche lernte ich dann zum ersten Mal meinen disziplinarischen Vorgesetzten Christoph Münzenmaier bei einem Kaffee kennen und konnte mich lange mit ihm über die aktuelle Lage, sowie die Vision des Projektes unterhalten. 
Weiterhin hatte ich dann den Auftrag von Dominik eine ursprünglich von BCG geschriebene Anleitung zum \ac{NCPA} mit neuen Screenshots und Slides zu überarbeiten, sodass dieses den neusten Stand der Applikation widerspiegelt. 
Zusätzlich bearbeitete ich noch einige notwendigen online Schulungen über Complience Grundlagen, Informationsschutz, Digital Identity, etc…, um damit dann auch das Onboarding final abzuschließen.

Abschließend zu Arbeitswoche wurde ich donnerstags nach der Arbeit auf ein Team-treffen Im Biergarten eingeladen, bei dem ich die schon bekannten, aber auch neue Kollegen näher kennenlernen durfte.


\subsection{KW16 14.04-2025 - 18.04.2025: Ostern}
Meine dritte Woche begann mit der Fortsetzung des Onepagers zur Standardization des Tech-Stacks für \ac{CBS}. 
Hierfür fand ein extensives Meeting mit mehreren ???, indem die Tools für verschiedene Bereiche (z.B. Data Science \& Machine Learning, CI/CD Chain oder Cloud Provides) diskutiert wurden. 
Nach abschließender Zusammenfassung der Meeting-Erkenntnisse überarbeitete ich den Tech Stack Onepager so, dass dieser in den folgenden Wochen von uns in verschiedensten Runden vorgestellt werden kann und damit unser übergreifendes Ziel der Standardisierung vorantreibt.

Ebenfalls viel beschäftigt habe ich mich in dieser Woche mit dem Sogenannten \ac{DOPM DC} Switch bei dem die Datasource einiger Daten aus dem \ac{NCPA} von \ac{OneGPM} (eine weitere Applikation der BMWGroup) nun von \ac{DOPM DC} bezogen werden. 
Für den switch musste ich Testcases erstellen, sodass die neuen Daten kontinuierlich und klar wiederholbar überprüft werden können.

Weiterhin wurde diese Woche mit meinen Kolleginnen Sibel und Sarah besprochen, dass wir zukünftig das regular release testing (Testing eines Systems kurz vor Release) für die Produktiv-, sowie die \ac{E2E} Umgebung des \ac{NCPA} übernehmen und dafür auch schon erste Termine eingestellt und wahrgenommen.

Unabhängig meiner eigentlichen Arbeit habe ich mich zusätzlich zu einigen Events der \ac{NAWI} (Nachwuchsinitative der BMWGroup) angemeldet und mich dann jeweils montags im Olympiapark, sowie mittwochs im Englischen Garten mit vielen anderen Praktikanten du Werksstudenten getroffen und ausgetauscht.


\subsection{KW17 21.04-2025 - 25.04.2025: PTSO Domain Planning}
Diese Woche begann mit einem sehr langen Meeting - dem PTSO Domain Planning für Cycle 5 mit einem anschließenden Vision Cycle 5 und Dependency talk Meeting. 
Einfach erklärt wird das Jahr in insgesamt 4 Cycles strukturiert (in den darauffolgenden Jahren wird weiter gezählt – daher Cycle 5), und jeder Cycle muss ausführlich geplant und besprochen werden. 
Weiterhin wurden im Vision Meeting der Ausblick auf die nächsten Cycles und im Dependency talk alle Abhängigkeiten der verschiedenen Produkte von allen anderen Produkten besprochen, sodass es hier zu keinen unnötigen Verzögerungen kommt. 
In der Restlichen Woche beschäftigte ich mich unter anderem mit der Erstellung von Progress Slides für das \ac{UAT} Wave 4 Event der zweiten Woche, da dieses den Abschluss des \ac{UAT} für das \ac{NCPA} darstellte und wir dem Team in einem großen Meeting in der darauffolgenden Woche eine Gesamtübersicht aller \acp{UAT} geben wollten. 
Außerdem startete ich einzelne Meetings mit anderen bekannten Praktikanten für allgemeine Verbesserungsvorschläge des Onboardings, um später den Onboarding Prozess weiterer Werksstudenten und Praktikanten zu vereinfachen.

Ebenfalls interessant war eine kleine Führung des BMW-Vierzylinders eines Freundes, dort als Praktikant tätig ist und einem anschließenden Essen dort, sowie eine offizielle, von der \ac{NAWI} organisierte Führung durch das BMW-Museum.

Abschließend zur Woche bearbeitete ich noch die wöchentlichen Meetings zum Regular Release Testing, aber auch weitere kleine Aufgaben, welche in den letzten Wochen noch überblieben.


\subsection{KW18 28.04-2025 - 02.05.2025: Vertretung}
In Woche fünf hatte sich Dominik Urlaub genommen, sodass ich in dieser Zeit als seine Vertretung gesetzt wurde. 
Diese Rolle wurde dann montags direkt auf die Probe gestellt, nachdem mich ein Tester aus dem Schweizer Markt mit einem Problem im \ac{NCPA} konfrontierte. 
Es wurde auf der \ac{INT}-Umgebung ein neuer Preis für ein Mini-Modell abgegeben, jedoch war dieser dann nicht auf den Folgesystemen wie der Website oder \ac{OFCO} zu sehen. 
Nach etwas Recherche in unserer \ac{AWS}-Datenbank fand ich heraus, dass die Preisänderung im \ac{NCPA} schon fehlerhaft abgeschickt wurde. 
In einer kurzen Meetingsession mit dem Schweizer Tester setzten wir zusammen einen neuen Preis, welchen ich dann manuell genehmigen und im \ac{AWS}-Lambda System an die weiteren Systeme pushen konnte. 
Da der Preis am darauffolgenden Tag korrekt in den Folgesystemen stand gehen wir von einem User Fehler aus. 
Neben dem Problem des Schweizer Markts und den damit einhergehenden Calls mit einigen Kollegen, durfte ich beim Sprintstart die in der Woche zuvor erstellten \ac{UAT} Wave 4 Progress Slides vorstellen und dem Team die aktuellen, wie auch die vergangenen KPI’s der Tests demonstrieren.

Im weiteren Verlauf der Woche hatte ich noch eine Handvoll Einzel-Gespräche mit unterschiedlichen Leuten der Domain, um das Tech-Stack standardization Thema weiter voranzutreiben, den Onepager auf einen fast fertigen Stand zu bringen und zu diskutieren, wie diese Standardisierungsmaßnahmen am besten durchgeführt werden sollten. 
Zugleich kam in einem der Regular Defect Meetings der Wunsch zur Erweiterung sowie weiteren Verbreitung unter den Teams des von mir erstellten Defect Templates.

Da der 1. Mai ein Feiertag ist und nahezu alle meiner Kollegen den Freitag als Brückentag nutzten konnte ich das Ende der Woche dazu nutzen eine veraltete \ac{E2E}-Testanleitung (Ursprünglich von BCG) für das \ac{NCPA} mit frischen Screenshots der UI, aber auch neuen Testschritten bestücken und die ganze Anleitung in unser XRAY Test Management System (eine Erweiterung der JIRA-Plattform, die zum Verwalten von Testprojekten entwickelt wurde) zu übertragen.

Abseits der Arbeit besuchte ich mittwochs wieder den Stammtisch im Englischen Garten der Nachwuchsinitiative.


\subsection{KW19 05.05-2025 - 09.05.2025: Erster Monat}
Woche sechs begann mit einem Review-Gespräch über meinen ersten Monat bei BMW seitens Dominik. 
Hier konnte ich ihm Feedback geben, bekam aber auch selbst sehr positives Feedback, sowie auch ein paar Tipps für meine restliche Zeit im Praktikum.
Außerdem konnte ich hier eine kleine Idee hinsichtlich der Vereinfachung des Onboardingprozesses an Dominik pitchen. 
Hierbei ging es um die Erstellung eines Onboarding-Chatbots mittels hauseigener BMW-KI, welche Zugriff auf verschiedenste Medien (z.B. Confluence, welches unsere Onboardinginformationen, sowie Abkürzungsverzeichnisse und weiteres beinhaltet) und KI-Modelle hat.
Anschließend dazu hatte ich Gespräche mit weiteren Personen die sich mit dem Thema Künstliche Intelligenz stark auseinandersetzen.

Hinsichtlich der Mitte der Woche beendete ich das Einfügen des zuvor fertiggestellten \ac{E2E}-Test-Manuals in das XRAY-system, sodass diese nun für den anstehenden Regular Release Testing Termine benutzt werden konnten.
Auch der oft erwähnte Standardization-Onepager bekam in dieser Woche noch einmal Aufmerksamkeit bei einem Meeting mit Merle und ??? (Die Book of Standards Person), welcher eine ähnliche Idee in der Domäne vorantreibt und somit Gespräche über den Zusammenschluss geführt wurden.
Der Onepager muss jedoch nicht nur gut aussehen, sondern letztendlich auch in jeder Applikation implementiert werden, sodass eine Standardisierung entsteht.
Dafür traf ich mich mit Dominik für ein Brainstorming zum IT-Leadership Workshop, indem wir beide zwei Stunden zugewiesen bekamen, um diesen den einzelnen Product-Ownern vorzustellen und final einzusetzen.


\subsection{KW20 12.05-2025 - 16.05.2025: IT Leadership Workshop vorbereitungen}
In dieser Woche wurden meine zuvor erstellten XRAY-Test zum ersten mal bei den Regular Release Testings ausgetestet.
Hierbei sind ein paar kleine Fehler aufgefallen welche im weiteren Verlauf der Woche noch verbessert wurden.
Weiterhin bekam ich eine neue Aufgabe zum Redesign des kompletten Confluence Bereichs für das Release Management der Domain.
Der Domain-Release Manager ist Dominik, der bis zu zu diesem Punkt noch alle Release Versionen und weitere Informationen zu jeder einzelnen Applikation (19 Stück) händisch gepflegt hatte.
Natürlich werden diese Informationen bei Vernachlässigung schnell rückständig und die neu definierte Struktur der Domain muss hier auch wiedergespiegelt werden.
Daher ist es sehr Sinnvoll aktuelle, sowie geplante Releases jeder einzelnen Applikation dynamisch aus ihren JIRA-Board zu entnehmen und darzustellen. 
In dieser Woche habe ich dann einen kleinen Prototypen für ein paar der Seiten vorgestellt um diesen dann in der daraufffolgenden Woche fertig zu stellen.

Ausser der Release Management Aufgabe beschäftigte ich mich diese Woche mit den letzten Vorbereitungen für den IT-Leadership Workshop, indem ich verschiedenste Vorlagen, Powerpoints und Konzepte ausarbeitete, welche unseren zweistündigenm Slot füllen.
Ebenfalls gegen mitte der Woche besuchte ich noch ein Info-Event der \ac{NAWI}, indem verschiedenste Nachwuchsprogramme (wie z.B. das Fastlane Programm für einen Berfusbegleitenden Master) vorgestellt und besprochen wurden, sodass ich mir ein Bild über eine mögliche weitere Zukunft bei BMW machen konnte.

Abseits der Arbeit besuchte ich Donnerstags in der Mittagspause einen alten Bekannten aus der SChulzeit in der Kantine des Werks, durch welches ich dann im Anschluss auch eine schnelle Führung bekam.



\subsection{KW21 19.05-2025 - 23.05.2025: Achte Woche}
\begin{itemize}
  \item NCPA User Manual Update
  \item IT Leadership WS
\end{itemize}

\subsection{KW22 26.05-2025 - 30.05.2025: Neunte Woche}


\subsection{KW23 02.06-2025 - 06.06.2025: Zehnte Woche}


\subsection{KW24 09.06-2025 - 13.06.2025: Elfte Woche}


\subsection{KW25 16.06-2025 - 20.06.2025: Zwölfte Woche}


\subsection{KW26 23.06-2025 - 27.06.2025: Dreizehnte Woche}


\subsection{KW27 30.06-2025 - 04.07.2025: Vierzehnte Woche}


\subsection{KW28 07.07-2025 - 11.07.2025: Fünfzehnte Woche}


\subsection{KW29 14.07-2025 - 18.07.2025: Sechzehnte Woche}


\subsection{KW30 21.07-2025 - 25.07.2025: Siebzehnte Woche}


\subsection{KW31 28.07-2025 - 01.08.2025: Achtzehnte Woche}


\subsection{KW32 04.08-2025 - 08.08.2025: Neunzehnte Woche}


\subsection{KW33 11.08-2025 - 15.08.2025: Zwanzigste Woche}


\subsection{KW34 18.08-2025 - 22.08.2025: Letzte Woche}


\subsection{KW35 25.08-2025 - 26.08.2025: Abschied}
