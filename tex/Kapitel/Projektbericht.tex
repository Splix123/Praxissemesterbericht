\section{Projektbericht}



\subsection{Abstract}
… ist eine Kurzfassung des gesamten Berichts. Der Text sollte nicht mehr als
15 Zeilen lang sein. Es ist schwierig, den Abstract vor dem Rest des Berichts zu
schreiben, da im Abstract das gesamte Projekt zusammengefasst wird, nicht
nur Zielstellung und Gliederung, sondern auch Ergebnisse und Fazit


\subsection{Einleitung}
… soll unbedarften Lesern klar machen, worum es geht, was das zu lösende
Problem ist (Vorgaben) und wie Sie es gelöst haben (keine Details!)


\subsection{Stand der Technik / Verwandte Arbeiten}
Hier ist ein guter Platz, vergleichbare Lösungen zu diskutieren bzw. zu
zeigen, dass Ihr Problem noch nicht gelöst worden ist. An dieser Stelle findet
man naturgemäß viele Zitate, mit denen Sie Bezug auf Vorarbeiten oder
vergleichbare Arbeiten nehmen müssen.


\subsection{Grundlagen}
Hier geben Sie Lesern eine Einführung in die grundlegenden Methoden
und Verfahren, die für das Verständnis Ihrer Lösung wichtig sind. In der
Regel zitieren Sie hier Literatur (z. B. firmeninterne Berichte, Lehrbücher,
Zeitschriftenartikel), in denen die betreffenden Verfahren im Detail
beschrieben sind.


\subsection{Lösung}
Hier beschreiben Sie die formelle Lösung Ihres Problems (z. B. unter
Zuhilfenahme von UML-Diagrammen). Diskutieren Sie nach Möglichkeit
Design-Entscheidungen, also welche Entwurfsalternativen es gab und warum es zu den jeweiligen
Entscheidungen gekommen ist.


\subsection{Implementierungskapitel}
Hier können Sie auf spezielle Aspekte Ihrer Implementierung eingehen.
Diskutieren Sie auch hier wieder nach Möglichkeit Umsetzungsentscheidungen,
also welche Implementierungalternativen es gab und warum es zu den jeweiligen Entscheidungen
gekommen ist. Mess- und Testergebnisse, sofern
nicht in einem eigenen Kapitel ausgeführt, sollten Sie hier mit unterbringen.
Durch Implementierung und Test zeigen Sie, dass die von Ihnen erarbeitete
formelle Lösung in der Praxis funktioniert.


\subsection{Ausblick, weiterführende Arbeiten - optional}
Schreiben Sie hier, wie Ihre Arbeit sinnvoll fortgeführt werden kann. Machen
Sie jedoch nur Vorschläge, die eng mit Ihrer Arbeit zusammen hängen.


\subsection{Zusammenfassung}
In der Zusammenfassung am Ende müssen Sie die von Ihnen erzielten (und
beschriebenen) Ergebnisse vollständig auflisten. Es wird sich niemand die
Mühe machen, das aus Ihrem Text herauszusuchen. Was hier nicht steht,
gehört nicht erkennbar zu Ihren Ergebnissen. Eine kurze Bewertung Ihrer
Ergebnisse (nicht Ihres Praktikums) ist auch aufzuführen.


\subsection{Literaturverzeichnis}
Es enthält sämtliche Literatur, die Sie für Ihre Arbeit verwendet und im
Text referenziert haben. Sehr wichtig ist dabei die Ankopplung an den Text:
Wenn Sie einen Sachverhalt, eine Formel oder Abbildung aus der Literatur
verwenden, setzen Sie hinter die Aussage im Text eine Referenz auf die
Quellenangabe im Literaturverzeichnis