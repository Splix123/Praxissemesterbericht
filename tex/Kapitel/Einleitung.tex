\section{Einleitung}
% //TODO: Besser schreiben
Das Praxissemester stellt einen zentralen Bestandteil meines Studiums dar und bietet die Möglichkeit, theoretisches Wissen in einem praktischen Arbeitsumfeld anzuwenden und zu vertiefen. 
Mein Praktikum habe ich bei der BMW Group absolviert, einem der weltweit führenden Premiumhersteller von Automobilen und Motorrädern. 
Während meines Praktikums war ich im Bereich IT-Projektmanagement tätig, mit dem Schwerpunkt auf Testmanagement. 
Ziel war es, die im Studium erworbenen Kenntnisse weiterzuentwickeln, praktische Erfahrungen in einem global agierenden Unternehmen zu sammeln und wertvolle Einblicke in die Abläufe und Herausforderungen des IT-Projektmanagements zu gewinnen.

\subsection{Firmenumfeld}
% //TODO: Besser schreiben
Die BMW Group ist ein international agierender Konzern mit Hauptsitz in München, Deutschland. 
Sie zählt zu den führenden Anbietern von Premiumfahrzeugen und Mobilitätsdienstleistungen. 
Neben den Automarken BMW, MINI und Rolls-Royce umfasst das Unternehmen auch ein breites Spektrum an Dienstleistungen, wie etwa Finanz- und Mobilitätslösungen. 
Mit weltweit über 100.000 Mitarbeitern und einem Netzwerk aus Produktionsstätten, Entwicklungszentren und Vertriebsgesellschaften ist die BMW Group ein Synonym für Innovation, Qualität und Nachhaltigkeit. 
Innerhalb der BMW Group nimmt die IT eine Schlüsselrolle ein. 
Sie treibt die Digitalisierung des Unternehmens voran, unterstützt die Entwicklung intelligenter Fertigungsprozesse und fördert innovative Technologien wie autonomes Fahren und vernetzte Mobilität. 
Der Bereich IT-Projektmanagement spielt dabei eine entscheidende Rolle, um sicherzustellen, dass umfangreiche IT-Projekte effizient und zielgerichtet umgesetzt werden.

\subsection{Zielsetzung des Praktikums}
% //TODO: Besser schreiben
Das Hauptziel meines Praktikums bestand darin, die Abläufe und Herausforderungen des Testmanagements in IT-Projekten kennenzulernen und aktiv mitzugestalten. Konkret umfasste dies unter anderem:
\begin{itemize}
  \item Die Planung und Organisation von Testprozessen in verschiedenen Projektphasen
  \item Die Zusammenarbeit mit interdisziplinären Teams, um Anforderungen zu analysieren und Tests entsprechend zu implementieren
  \item Die Sicherstellung der Qualität durch die Durchführung und Dokumentation von Testfällen
  \item Die Identifikation und das Management von Risiken, die während der Testphase auftreten können
\end{itemize}
Durch die aktive Mitarbeit in laufenden Projekten konnte ich nicht nur meine fachlichen Kompetenzen erweitern, sondern auch Soft Skills wie Teamarbeit, Kommunikation und Problemlösungsfähigkeit stärken. 
Ein weiteres Ziel war es, ein tiefgehendes Verständnis für die Arbeitsweise eines global agierenden Unternehmens wie der BMW Group zu erlangen und die Bedeutung von IT-Prozessen in einem dynamischen Umfeld zu erfassen.
